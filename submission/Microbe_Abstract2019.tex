\documentclass[11pt,]{article}
\usepackage{lmodern}
\usepackage{amssymb,amsmath}
\usepackage{ifxetex,ifluatex}
\usepackage{fixltx2e} % provides \textsubscript
\ifnum 0\ifxetex 1\fi\ifluatex 1\fi=0 % if pdftex
  \usepackage[T1]{fontenc}
  \usepackage[utf8]{inputenc}
\else % if luatex or xelatex
  \ifxetex
    \usepackage{mathspec}
  \else
    \usepackage{fontspec}
  \fi
  \defaultfontfeatures{Ligatures=TeX,Scale=MatchLowercase}
\fi
% use upquote if available, for straight quotes in verbatim environments
\IfFileExists{upquote.sty}{\usepackage{upquote}}{}
% use microtype if available
\IfFileExists{microtype.sty}{%
\usepackage{microtype}
\UseMicrotypeSet[protrusion]{basicmath} % disable protrusion for tt fonts
}{}
\usepackage[margin=1.0in]{geometry}
\usepackage{hyperref}
\hypersetup{unicode=true,
            pdfborder={0 0 0},
            breaklinks=true}
\urlstyle{same}  % don't use monospace font for urls
\usepackage{graphicx,grffile}
\makeatletter
\def\maxwidth{\ifdim\Gin@nat@width>\linewidth\linewidth\else\Gin@nat@width\fi}
\def\maxheight{\ifdim\Gin@nat@height>\textheight\textheight\else\Gin@nat@height\fi}
\makeatother
% Scale images if necessary, so that they will not overflow the page
% margins by default, and it is still possible to overwrite the defaults
% using explicit options in \includegraphics[width, height, ...]{}
\setkeys{Gin}{width=\maxwidth,height=\maxheight,keepaspectratio}
\IfFileExists{parskip.sty}{%
\usepackage{parskip}
}{% else
\setlength{\parindent}{0pt}
\setlength{\parskip}{6pt plus 2pt minus 1pt}
}
\setlength{\emergencystretch}{3em}  % prevent overfull lines
\providecommand{\tightlist}{%
  \setlength{\itemsep}{0pt}\setlength{\parskip}{0pt}}
\setcounter{secnumdepth}{0}
% Redefines (sub)paragraphs to behave more like sections
\ifx\paragraph\undefined\else
\let\oldparagraph\paragraph
\renewcommand{\paragraph}[1]{\oldparagraph{#1}\mbox{}}
\fi
\ifx\subparagraph\undefined\else
\let\oldsubparagraph\subparagraph
\renewcommand{\subparagraph}[1]{\oldsubparagraph{#1}\mbox{}}
\fi

%%% Use protect on footnotes to avoid problems with footnotes in titles
\let\rmarkdownfootnote\footnote%
\def\footnote{\protect\rmarkdownfootnote}

%%% Change title format to be more compact
\usepackage{titling}

% Create subtitle command for use in maketitle
\newcommand{\subtitle}[1]{
  \posttitle{
    \begin{center}\large#1\end{center}
    }
}

\setlength{\droptitle}{-2em}

  \title{}
    \pretitle{\vspace{\droptitle}}
  \posttitle{}
    \author{}
    \preauthor{}\postauthor{}
    \date{}
    \predate{}\postdate{}
  
\usepackage{booktabs}
\usepackage{longtable}
\usepackage{array}
\usepackage{multirow}
\usepackage[table]{xcolor}
\usepackage{wrapfig}
\usepackage{float}
\usepackage{colortbl}
\usepackage{pdflscape}
\usepackage{tabu}
\usepackage{threeparttable}
\usepackage{threeparttablex}
\usepackage[normalem]{ulem}
\usepackage{makecell}

\usepackage{helvet} % Helvetica font
\renewcommand*\familydefault{\sfdefault} % Use the sans serif version of the font
\usepackage[T1]{fontenc}

\usepackage[none]{hyphenat}

\usepackage{setspace}
\doublespacing
\setlength{\parskip}{1em}

\usepackage{lineno}

\usepackage{pdfpages}

\begin{document}

\linenumbers
\textbf{Evaluation of classification pipelines that predict colorectal
cancer progression with microbiota-associated biomarkers}

Begüm D. Topçuoğlu\({^1}\), Jenna Wiens\({^2}\), Patrick D.
Schloss\textsuperscript{1\(\dagger\)}

As gut microbiome field continues to grow, there will be an
ever-increasing demand for reproducible machine learning methods to
determine association of the microbiome with a continuous or categorical
phenotype of interest. Currently, the use of machine learning in
microbiome literature lack clarity over the training, validation and
testing of the models used. There is a need to properly implement good
machine learning practices to generate reproducible and robust models.

Recently, there is an interest in using machine learning to predict
colorectal cancer progression with microbiota-associated biomarkers.
Colorectal cancer is one of the leading cause of death among cancers in
the United States. Each person in the industrialized world has on
average a one-in-twenty chance of developing colorectal cancer (CRC)
(1--3). Colonoscopy as a screening tool is effective, however it is
invasive, expensive and have a low rate of patient adherence. Therefore,
gut microbiome-based biomarkers emerged as a non-invasive screening
method.

In this study, classification models that use human hemoglobin levels
and bacterial population abundances in the stool were used to predict
colorectal disease status as screen-relevant colonic growth or not.
Training, validation and testing pipelines were established for
L2-regularized Logistic Regression, L1 and L2 Linear Suppor Vector
Machines (SVM), Radial Basis Function SVM, Decision Tree, Random Forest
and XGBoost classifiers. The generalization and prediction performance
of these classifiers were evaluated and each classifier was examined for
its reproducibility, robustness and susceptibility to overfitting.
L2-regularized Logistic Regression had a mean AUC of 0.68 +/- 0.04, L1
Linear SVM had a mean AUC of 0.76 +/- 0.05, L2 Linear SVM had a mean AUC
of 0.68 +/- 0.05 and Radial Basis Function SVM had a mean AUC of 0.69
+/- 0.05. Decision Tree had a mean AUC of 0.7 +/- 0.05, Random Forest
had a mean AUC of 0.76 +/- 0.06 and XGBoost had a mean AUC of 0.76 +/-
0.04. Tree-based models were less susceptible to overfiting and in
general had higher sensitivity and specificity for colonic
screen-relevant growth.

\newpage

\subsection{References}\label{references}

\hypertarget{refs}{}
\hypertarget{ref-seer_2016}{}
1. \textbf{Howlader N KM \textnormal{Noone AM}}. SEER cancer statistics
review, 1975-2013, (national cancer institute. bethesda, md).

\hypertarget{ref-street_colorectal_nodate}{}
2. \textbf{Street W}. Colorectal cancer facts \& figures 2017-2019 40.

\hypertarget{ref-weir_past_2015}{}
3. \textbf{Weir HK}, \textbf{Thompson TD}, \textbf{Soman A},
\textbf{MÞller B}, \textbf{Leadbetter S}. 2015. The past, present, and
future of cancer incidence in the united states: 1975 through 2020.
Cancer \textbf{121}:1827--1837.
doi:\href{https://doi.org/10.1002/cncr.29258}{10.1002/cncr.29258}.


\end{document}
